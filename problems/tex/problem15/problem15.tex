
%% bare_adv.tex
%% V1.4b
%% 2015/08/26
%% by Michael Shell
%% See: 
%% http://www.michaelshell.org/
%% for current contact information.
%%
%% This is a skeleton file demonstrating the advanced use of IEEEtran.cls
%% (requires IEEEtran.cls version 1.8b or later) with an IEEE Computer
%% Society journal paper.
%%
%% Support sites:
%% http://www.michaelshell.org/tex/ieeetran/
%% http://www.ctan.org/pkg/ieeetran
%% and
%% http://www.ieee.org/

%%*************************************************************************
%% Legal Notice:
%% This code is offered as-is without any warranty either expressed or
%% implied; without even the implied warranty of MERCHANTABILITY or
%% FITNESS FOR A PARTICULAR PURPOSE! 
%% User assumes all risk.
%% In no event shall the IEEE or any contributor to this code be liable for
%% any damages or losses, including, but not limited to, incidental,
%% consequential, or any other damages, resulting from the use or misuse
%% of any information contained here.
%%
%% All comments are the opinions of their respective authors and are not
%% necessarily endorsed by the IEEE.
%%
%% This work is distributed under the LaTeX Project Public License (LPPL)
%% ( http://www.latex-project.org/ ) version 1.3, and may be freely used,
%% distributed and modified. A copy of the LPPL, version 1.3, is included
%% in the base LaTeX documentation of all distributions of LaTeX released
%% 2003/12/01 or later.
%% Retain all contribution notices and credits.
%% ** Modified files should be clearly indicated as such, including  **
%% ** renaming them and changing author support contact information. **
%%*************************************************************************


% *** Authors should verify (and, if needed, correct) their LaTeX system  ***
% *** with the testflow diagnostic prior to trusting their LaTeX platform ***
% *** with production work. The IEEE's font choices and paper sizes can   ***
% *** trigger bugs that do not appear when using other class files.       ***                          ***
% The testflow support page is at:
% http://www.michaelshell.org/tex/testflow/


% IEEEtran V1.7 and later provides for these CLASSINPUT macros to allow the
% user to reprogram some IEEEtran.cls defaults if needed. These settings
% override the internal defaults of IEEEtran.cls regardless of which class
% options are used. Do not use these unless you have good reason to do so as
% they can result in nonIEEE compliant documents. User beware. ;)
%
%\newcommand{\CLASSINPUTbaselinestretch}{1.0} % baselinestretch
%\newcommand{\CLASSINPUTinnersidemargin}{1in} % inner side margin
%\newcommand{\CLASSINPUToutersidemargin}{1in} % outer side margin
%\newcommand{\CLASSINPUTtoptextmargin}{1in}   % top text margin
%\newcommand{\CLASSINPUTbottomtextmargin}{1in}% bottom text margin

%
\documentclass[12pt, a4paper]{article}
%\documentclass[10pt,conference,compsoc,onecolumn]{IEEEtran}
%\def\IEEEcompsocdiamondline{\vspace{-2cm}}

\usepackage{listings}
\lstset{
	basicstyle=\ttfamily\linespread{1.1}\selectfont,
	breaklines=true,
	emph={
		sort, subsort, sorts, op, ops, eq, ceq, var, vars,
		protecting, extending, including, rl, crl,
		mod, is, endm, if, implies, fmod, endfm,
		and, or, else, fi, then, not, open, red, reduce, close,
		load, erewrite 
	},
	emphstyle={\color{black}\bfseries},
%	aboveskip=0.8em,
%	belowskip=0.8em
}
	
\usepackage[linesnumbered,ruled,vlined,noend]{algorithm2e}
\usepackage{soul}
\usepackage{multirow}
\usepackage{adjustbox}
\usepackage{xcolor}
\usepackage{amsmath,amssymb,amsfonts}
\usepackage{amsthm}
\usepackage{algorithmic}
\usepackage{graphicx}
\usepackage{hyperref}
\usepackage{url}
\usepackage{physics}
\definecolor{leafgreen}{rgb}{0.01, 0.75, 0.24}
\usepackage{pifont}% http://ctan.org/pkg/pifont
\newcommand{\cmark}{\ding{51}}%
\newcommand{\xmark}{\ding{55}}%

\newtheorem{theorem}{Theorem}
\newtheorem{lemma}[theorem]{Lemma}
\newtheorem{remark}{Remark}
\newtheorem{problem}{Problem}
%\newtheorem{proof}{Proof}
\newcommand{\comp}{\mathrel{;}}
\newcommand{\lmerge}{\mathrel{\llfloor}}
\newcommand{\rel}[1]{\mathrel{#1}}
\newcommand{\rmx}[1]{\mathrm{#1}}
\newcommand{\larrow}{\longrightarrow}
\newcommand{\under}{\underline}
\allowdisplaybreaks

\title{Problem 15}
\date{\vspace{-5ex}}

\begin{document}

\maketitle

\begin{problem}
$\forall L \in \mathtt{NatList}, \rmx{diff}(L, L) = nil$.
\end{problem}
\begin{proof}
By structural induction on $L$.
\begin{description}
\item[(1) Base case]~\\
\noindent
What to show: $\quad \rmx{diff}(nil, nil) = nil$.
\begin{align*}
%%% LHS
\under{\rmx{diff}(nil, nil)}
	&\larrow nil \tag{by diff1}
\end{align*}

\item[(2) Induction case]~\\
What to show: $\quad \rmx{diff}(x \mid l, x \mid l) = nil$ \\
Induction hypothesis: $\quad \rmx{diff}(l, l) = nil$  \\
where $x \in \mathtt{PNat}$ and $l \in \mathtt{NatList}$.\\
We use case splitting for our proofs as follows: \\
\textbf{Case 1}: $\rmx{has}(l, x) = true$
\begin{align*}
%%% LHS
\under{\rmx{diff}(x \mid l, x \mid l)}
	&\larrow \rmx{drop}(\under{\rmx{diff}(x \mid l, l)}, x) \tag{by Problem 14} \\
	&\larrow \rmx{drop}(\rel{\rmx{if}} \under{\rmx{has}(l, x)} \rel{\rmx{then}} \rmx{diff}(l, l) \rel{\rmx{else}} (x \mid \rmx{diff}(l, l)) \rel{\rmx{fi}}, x) \tag{by diff2} \\
	&\larrow \rmx{drop}(\under{\rel{\rmx{if}} true \rel{\rmx{then}} \rmx{diff}(l, l) \rel{\rmx{else}} (x \mid \rmx{diff}(l, l)) \rel{\rmx{fi}}}, x) \tag{by case splitting} \\
	&\larrow \rmx{drop}(\under{\rmx{diff}(l, l)}, x) \tag{by if1} \\
	&\larrow \under{\rmx{drop}(nil, x)} \tag{by IH} \\
	&\larrow nil \tag{by drop1} \\
\end{align*}

\textbf{Case 2}: $\rmx{has}(l, x) = false$
\begin{align*}
%%% LHS
%%% LHS
\under{\rmx{diff}(x \mid l, x \mid l)}
	&\larrow \rmx{drop}(\under{\rmx{diff}(x \mid l, l)}, x) \tag{by Problem 14} \\
	&\larrow \rmx{drop}(\rel{\rmx{if}} \under{\rmx{has}(l, x)} \rel{\rmx{then}} \rmx{diff}(l, l) \rel{\rmx{else}} (x \mid \rmx{diff}(l, l)) \rel{\rmx{fi}}, x) \tag{by diff2} \\
	&\larrow \rmx{drop}(\under{\rel{\rmx{if}} false \rel{\rmx{then}} \rmx{diff}(l, l) \rel{\rmx{else}} (x \mid \rmx{diff}(l, l)) \rel{\rmx{fi}}}, x) \tag{by case splitting} \\
	&\larrow \rmx{drop}(x \mid \under{\rmx{diff}(l, l)}, x) \tag{by if2} \\
	&\larrow \under{\rmx{drop}(x \mid nil, x)} \tag{by IH} \\
	&\larrow\ \rel{\rmx{if}} \under{(x = x)} \rel{\rmx{then}} \rmx{drop}(nil, x) \rel{\rmx{else}} (x \mid \rmx{drop}(nil, x)) \rel{\rmx{fi}} \tag{by drop2} \\
	&\larrow \under{\rel{\rmx{if}} true \rel{\rmx{then}} \rmx{drop}(nil, x) \rel{\rmx{else}} (x \mid \rmx{drop}(nil, x)) \rel{\rmx{fi}}} \tag{by equality} \\
	&\larrow \under{\rmx{drop}(nil, x)} \tag{by if1} \\
	&\larrow nil \tag{by drop1}
\end{align*}
\end{description}

\end{proof}
\end{document}